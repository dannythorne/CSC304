
\documentclass{article}

\setlength{\parindent}{0pt}

\begin{document}
{\Large\bf Heapify}

{\large\it Asymptotic Run-Time Complexity Analysis}

\vspace{1pc}
Given a list of $n$ items in an array, it is easy to convert the list into a
heap in two different ways, top-down and bottom-up. On casual consideration,
the two approaches may appear to cost the same. On closer analysis, the
bottom-up approach is seen to be asymptotically more efficient.

\vspace{1pc}
For the following analysis,
assume a full tree of height $h$,
so $n=2^{h+1}-1$ and $h=\log_2(n+1) - 1$
where $n$ is the number of nodes in the tree.
(The general case of heapifying a complete tree has a time complexity that is
squeezed between those of two full trees.)

\vspace{1pc}
{\large\bf Top-Down Heapify}

\vspace{1pc}
The heap is grown from left to right in the array, one entry at a time.
Each new entry from the remaining portion of the array is added to the heap
by shifting it upward into the heap as needed to acheive the heap property.

\vspace{1pc}
The $k^{th}$ node added to the heap may need to be shifted up as many as
$\log_2(k)$ times, so an upper bound on number of shifts is
\[
  \sum_{k=1}^{n} \log_2(k)
=
  \log_2\!\!\left(\prod_{k=1}^{n} k\right)
=
  \log_2(n!)
\]
and, as we showed earlier in the semester, $\log(n!)=\Theta\big(n\log(n)\big)$.

\vspace{1pc}
{\large\bf Bottom-Up Heapify}

\vspace{1pc}
Bottom-up heapify proceeds as follows

\vspace{1pc}
1. first heapify all the little trees of height 1 at the bottom

2. then heapify the trees of height 2 up one from bottom

3. then heapify the trees of height 3 up one from the previous step

\begin{tabular}{ll}
\vdots & and so on \\
\end{tabular}

\vspace{0.5pc}
$h$. until heapifying the one final tree of height $h$

\vspace{1pc}
Here's a list of the number of trees at each height:
\begin{center}
\begin{tabular}{c|c}
height  & num trees \\
\hline
%0      & $2^{h}$ \\
 1      & $2^{h-1}$ \\
 2      & $2^{h-2}$ \\
 \vdots & \vdots    \\
 $h-1$  & 2         \\
 $h$    & 1         \\
\end{tabular}
\end{center}

For each of the $2^{h-k}$ trees of height $k$, up to $k$ swaps will be required
to heapify it (since both of its subtrees are already heaps), so that's an
upper bound of $k2^{h-k}$ swaps for the trees of height $k$. Summing all those
products for
$1\le k\le h$ gives
\[
  \sum_{k=1}^{h} k2^{h-k}
=
  1\cdot2^{h-1} + 2\cdot2^{h-2} + \cdots + (h-1)2^1 + h2^0
\]

To figure out this sum, multiply it by 2
\[
  2\sum_{k=1}^{h} k2^{h-k}
=
  1\cdot2^{h} + 2\cdot2^{h-1} + 3\cdot2^{h-2} + \cdots + (h-1)2^2 + h2^1
\]
and take the difference
\[
  2\sum_{k=1}^{h} k2^{h-k}
-
  \sum_{k=1}^{h} k2^{h-k}
=
  1\cdot2^{h} + 1\cdot2^{h-1} + 1\cdot2^{h-2} + \cdots + 1\cdot2^2 + 1\cdot2^1 - h2^0
\]
then rewrite
\[
  \sum_{k=1}^{h} k2^{h-k}
=
  2^{h} + 2^{h-1} + 2^{h-2} + \cdots + 2^2 + 2^1 - h2^0
=
  \left(\sum_{k=1}^{h} 2^{k}\right) - h
\]
and note that
\[
  \left(\sum_{k=1}^{h} 2^{k}\right)
= 
  2^1 + 2^2 + \cdots + 2^h
\]
and
\[
  2\left(\sum_{k=1}^{h} 2^{k}\right)
= 
  2^2 + 2^3 + \cdots + 2^{h+1}
\]
and the difference gives (using the same technique as with the original sum)
\[
  2\left(\sum_{k=1}^{h} 2^{k}\right)
  -\left(\sum_{k=1}^{h} 2^{k}\right)
=
  \sum_{k=1}^{h} 2^{k}
=
  2^{h+1} - 2
\]
so, finally,
\[
  \sum_{k=1}^{h} k2^{h-k}
=
\quad
  \left(\sum_{k=1}^{h} 2^{k}\right) - h
=
  2^{h+1} - 2 - h
\quad
=
  \Theta(2^h)
=
  \Theta(n).
\]

Note that Sedgewick gets
\[
  \sum_{k=1}^{h} (k-1)2^{h-k}
=
  2^{h} - 1 - h.
\]
Either I'm missing something or his $k-1$ is a typo.
It's not important in terms of the asymptotic analysis, though.
The run-time complexity of bottom-up heapify is shown to be \textbf{linear}
either way.

\vspace{1pc}
{\large\bf Exercise}

Using our understanding of $\log_2(n!)$ in the top-down analysis is unecessary.
An analysis that is similar to what we did in the bottom-down case applies in
the top-down case as well and leads to
this
\[
  \sum_{k=1}^{h} k2^k
\]
expression for the upper bound which appears deceptively similar to the one we
found in the bottom-down case. Using the same technique as in the bottom down
case, show that this sum is proportional to $n\log(n)$.

\end{document}
