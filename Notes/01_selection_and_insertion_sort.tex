
\documentclass{article}

\setlength{\parindent}{0pc}
\setlength{\parskip}{1pc}

\begin{document}

{\Large\bf Insertion Sort and Selection Sort}

Sorting algorithms are of fundamental importance and make a natural place to start the study of algorithms.
The most common type of sorting algorithm works by systematically comparing
entries and repositioning them according to their relative size
until they are all in order.
They are called
\textit{comparison sorts}.
A couple of common elementary comparison sorts are
\textit{selection sort}
and
\textit{insertion sort}.
These two algorithms are described below and some preliminary analysis
is given. The discussion assumes the algorithms are being applied to
a list of $n$ items indexed starting from $0$,
so the last entry in the list is indexed by $n-1$.
References to the
$i^{\mathrm{th}}$
entry or
$i^{\mathrm{th}}$
step mean the entry or step indexed by $i$.

{\bf Selection Sort ---}
Traverse the entries from $0$ to $n-2$.
At the
$i^{\mathrm{th}}$
entry,
select from the remaining items the one that should go
there and swap it into place.
When sorting in ascending order, this means finding the
minimum of the remaining entries.

Here is a tiny example:

\begin{tabular}{ccccl}
\framebox{3} &           2  &           4  &           1  & \\
          1  & \framebox{2} &           4  &           3  & \\
          1  &           2  & \framebox{4} &           3  & \\
          1  &           2  &           3  &           4  & \\
\end{tabular}

TODO: Picture of a general $i^{\mathrm{th}}$ step.

At each of the $n-1$ steps, finding the minimum of the remaining
$n-i-1$ items requires $n-i-1$ comparisons, so the total number
of comparisons required to sort $n$ items with selection sort is
\[
  \sum_{i=0}^{n-2} n-i-1
\;=\;
  \sum_{i=1}^{n-1} i
\;=\;
  \frac{1}{2}n(n-1)
\;=\;
  \frac{1}{2}n^2-\frac{1}{2}n
\]
Each step requires at most one swap. A swap requires 3 assignments.
So selection sort requires at most $3n$ assigments to sort $n$ items.

{\bf Insertion Sort ---}
Traverse the entries from $1$ to $n-1$.
At the
$i^{\mathrm{th}}$
entry,
insert the item from there into its correct spot to the left, shifting
items to the right to make space for it.

Here is a tiny example:

\begin{tabular}{ccccl}
          3  & \framebox{2} &           4  &           1  & \\
          2  &           3  & \framebox{4} &           1  & \\
          2  &           3  &           4  & \framebox{1} & \\
          1  &           2  &           3  &           4  & \\
\end{tabular}

At each of the $n-1$ steps, an average of
$\frac{1}{2}i$
comparisons and
$\frac{1}{2}i$
assignments are performed,
so the total number of each required for insertion sort
on average is
\[
  \frac{1}{2}\sum_{i=1}^{n-1} i
\;=\;
  \frac{1}{2}\left( \frac{1}{2}n(n-1)\right)
\;=\;
  \frac{1}{4}n(n-1)
\;=\;
  \frac{1}{4}n^2-\frac{1}{4}n
\]
What is the best case and worse case for insertion sort?


\end{document}
