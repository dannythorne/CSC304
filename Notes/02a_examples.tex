
\documentclass{article}

\begin{document}

{\Large\bf Asymptotic Analysis Examples}

\vspace{1pc}
Determine the asymptotic relationship between $f$ and $g$.

\begin{itemize}
\item
$f(n) = \frac{1}{2}n^2 - 3n$

$g(n) = n^2$

We want to find a $c>0$ and $n_0>0$ such that
\[
  \frac{1}{2}n^2 - 3n \le cn^2
\]
for all $n\ge n_0$.
Dividing through by $n^2$, we see
\[
  \frac{1}{2} - \frac{3}{n} \le c
\]
so we can choose $n_0=1$ and then
\[
  \Big(
  0 <
  \Big)
  \quad
  c = 1
  \quad
  \left(
  \ge
  \frac{1}{2} - \frac{3}{n_0}
  = -\frac{5}{2}
  \right)
\]
to satisify the definition of
\[
  f(n) = O\big(g(n)\big).
\]

Then we try to find a
$c>0$ and $n_0>0$ such that
\[
  n^2 \le c\left(\frac{1}{2}n^2 - 3n\right)
\]
for all $n\ge n_0$.
Dividing through by $n^2$, we see
\[
  1 \le c\left(\frac{1}{2} - \frac{3}{n}\right).
\]
If we choose something like $n_0=12$, then
\[
  1 \le c\left(\frac{1}{2} - \frac{3}{n_0}\right)
\Longrightarrow
  1 \le \frac{1}{4}c
\Longrightarrow
  c \ge 4
\]
so choosing $c=4$ satisfies the definition of
\[
  g(n) = O\big(f(n)\big).
\]

We've shown both
$f=O(g)$
and
$g=O(f)$
so that means
$f=\Theta(g)$.

\vspace{1pc}
\item
$f(n) = n^3 - 5n$

$g(n) = 2n^2 + 25n$

We'll start by trying to satisfy the definition of $f=O(g)$
by trying to find a $c>0$ and $n_0>0$ such that
\[
  n^3 - 5n \le c\left(2n^2 + 25n\right)
\]
for all $n\ge n_0$. Dividing through by $n^3$, we have
\[
  1 - \frac{5}{n^2} \le c\left(\frac{2}{n} + \frac{25}{n^2}\right).
\]
By casual inspection we know that no matter what value is chosen
for $c$ it is always possible to make $n$ big enough to violate
the inequality, but let's be more formal.
For $n\ge3$,
\[
  1 - \frac{5}{n^2} > \frac{1}{2},
\]
so to show that the definition cannot be satisfied it is sufficient
to consider
\[
  \frac{1}{2} \le c\left(\frac{2}{n} + \frac{25}{n^2}\right).
\]
Multiplying through by $n^2$ now gives
\[
  \frac{1}{2}n^2 \le c\left(2n + 25\right)
\quad
\Longrightarrow
\quad
  \frac{n^2}{4n+50} \le c.
\]
We can use this to show, given any $c$, how to choose $n$ large enough to violate the original inequality.
Since, for $n>12$
\[
  \frac{n^2}{4n+50}
  >
  \frac{n^2}{8n}
  =
  \frac{1}{8}n,
\]
choosing $n\ge8c$ means
\[
  \frac{n^2}{4n+50} > c
\]
and, in turn,
\[
  1 - \frac{5}{n^2} > \frac{1}{2} > c\left(\frac{2}{n} + \frac{25}{n^2}\right).
\]
and, finally,
\[
  n^3 - 5n > c\left(2n^2 + 25n\right).
\]
So, $f\ne O(g)$.
That means $f=\omega(g)$.

\vspace{1pc}
\item $f(n) = 6n^3$, $g(n) = n^2$

\[
  6n^3 \le cn^2
\Longrightarrow
  6n \le c
\Longrightarrow
  \not\exists c, n_0 \mbox{ s.t. } 6n^3 \le cn^2 \mbox{ for } n>n_0
\]
so $f(n)\ne O(g(n))$.

\[
  6n^3 \ge cn^2
\Longrightarrow
  6n > c
\Longrightarrow
  n > \frac{c}{6}
\]
so for any $c$, choosing $n > \frac{c}{6}$ makes $f(n) > g(n)$.
That means $f=\omega(g)$ or $g = o(f)$.

\vspace{1pc}
\item $f(n) = n^{1/2}$, $g(n) = n^{2/3}$

\[
  \lim_{n\rightarrow\infty}\frac{n^{1/2}}{n^{2/3}}
=
  \lim n^{1/2-2/3}
=
  \lim n^{-1/6}
=
  0
\]
so $f(n)=o(g(n))$.

Note that, correspondingly,
\[
  \lim_{n\rightarrow\infty}\frac{n^{2/3}}{n^{1/2}}
\Longrightarrow
  \lim n^{1/6} = \infty.
\]

\vspace{1pc}
\item $f(n) = 100n + \log(n)$, $g(n) = n + (\log(n))^2$

\[
  \lim_{n\rightarrow\infty} \frac{100n + \log(n)}{n + (\log(n))^2}
=
  \lim_{n\rightarrow\infty} \frac{100 + \frac{1}{n}}{1 + 2(\log(n))\frac{1}{n}}
=
  \lim_{n\rightarrow\infty} \frac{100n + 1}{n + 2\log(n)}
=
  \lim_{n\rightarrow\infty} \frac{100}{1 + \frac{2}{n}} = 100
\]
so $f(n) = O(g(n))$.

\[
  \lim_{n\rightarrow\infty} \frac{n + (\log(n))^2}{100n + \log(n)}
=
  \frac{1}{100}
\]
so $f(n) = \Omega(g(n))$.

Together, that means $f(n) = \Theta(g(n))$.


\end{itemize}

\end{document}
